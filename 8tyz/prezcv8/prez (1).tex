% Valentino Vranic
% Metody inzinierskej prace 2012/13

\documentclass{beamer}

%\usetheme{Warsaw}
%\usetheme{Antibes}
\usetheme{JuanLesPins}
%\usetheme{Goettingen}

\usecolortheme{rose}
%\usecolortheme{dolphin}
%\usecolortheme{rose}
% http://deic.uab.es/~iblanes/beamer_gallery/index_by_color.html
%\usecolortheme{beaver}

%\useoutertheme[]{sidebar}

\setbeamercovered{transparent}

\usepackage[slovak]{babel}
\usepackage[T1]{fontenc}
\usepackage[utf8]{inputenc}
\usepackage{url}

\usepackage{listings}

\lstset{language=C++,basicstyle=\fontsize{8}{9.6}\selectfont,showstringspaces=false,columns=fullflexible,identifierstyle=\ttfamily,keywordstyle=\bfseries,showstringspaces=false,columns=fullflexible}
%\lstset{language=C,basicstyle=\fontsize{10.5}{12.6}\selectfont,identifierstyle=\ttfamily,keywordstyle=\bfseries,showstringspaces=false,columns=fixed}

\def\BibTeX{\textsc{Bib}\kern-.08em\TeX} 

\newcommand{\footcite}[1]{\footnote{\tiny #1}}
\newcommand{\umlet}{.5}
\newcommand{\emp}[1]{\textit{\alert{#1}}}
\newcommand{\kw}[1]{\mbox{\textbf{#1}}}
\newcommand{\id}[1]{\texttt{#1}}
\newcommand{\stl}{\guillemotleft}
\newcommand{\str}{\guillemotright}

\newcommand{\lsti}{\lstinline[basicstyle=\fontsize{10.5}{12.1}\selectfont]}

\newcommand{\ssection}[1]{
	\section{#1}
	\begin{frame}[fragile=singleslide]\frametitle{}
	\Huge #1
	\end{frame}
}

\newcommand{\ssectionn}[1]{
	\section*{#1}
	\begin{frame}[fragile=singleslide]\frametitle{}
	\Huge #1
	\end{frame}
}

\newenvironment{program}{\begin{beamercolorbox}[rounded=true,shadow=true]{block body}\vspace{-4mm}}{\vspace{-2mm}\end{beamercolorbox}}

\setbeamercolor{fvystup}{fg=white,bg=black}
\newenvironment{vystup}{\begin{beamercolorbox}[rounded=true,shadow=true]{fvystup}}{\end{beamercolorbox}}

\newenvironment{poznamka}{\begin{beamercolorbox}[rounded=true,shadow=false]{block body}}{\end{beamercolorbox}}

\setbeamertemplate{footline}[page number]
{
%\insertpagenumber
%\begin{beamercolorbox}{section in head/foot}
%\vskip2pt\insertnavigation{\paperwidth}\vskip2pt
%\end{beamercolorbox}%
}



\author{Petra Miková}
%\url{www.fiit.stuba.sk/~vranic}, \url{vranic@fiit.stuba.sk}}
%{\tiny \url{www.fiit.stuba.sk/~vranic}, \url{vranic@fiit.stuba.sk}}
\institute{
	Fakulta informatiky a informačných technológií\\
	Slovenská technická univerzita v Bratislave}

\subtitle{\vspace{3mm} Metódy inžinierskej práce 2019/2020}

\title{Gamifikácia a seriózne hry v medicínskom vzdelávaní
}

\date{\footnotesize 10. november 2022}




\begin{document}

\begin{frame}[fragile=singleslide]
\titlepage
\end{frame}


\begin{frame}[fragile=singleslide]\frametitle{Motivácia}
Vo svete už od pradávna ľudstvo potrebovalo lekárov, v minulosti skôr ľudí označovaných ako liečiteľov, až pokým nebol zavedený termín lekár. Rozdiel medzi dnešnými lekármi a liečiteľmi pred pár stovkami rokov je však ten, že v dnešnej dobe lekár svoje poznatky nemusí nadobúdať formou pozorovania a skúšania rôznych liečiv a prístupov k ochoreniam, ale vie tieto poznatky nadobudnúť z už publikovaných odborných učebníc a materiálov. V tejto prezentácii sa pozrieme na to, ako vie informatika pridať na efektivite tohto vzdelávania.
\end{frame}


\begin{frame}[fragile=singleslide]\frametitle{Prehľad prezentácie}
\tableofcontents
\end{frame}


\section{Hlavný problém}
% príkaz \ssection by vytvoril zvláštný slajd s názvom časti - v krátkych prezentáciách to prekáža, lebo oberá o čas

\begin{frame}[fragile=singleslide]\frametitle{Hlavný problém}
V tejto časti sa pozrieme na problém neefektívneho memorovania a nedostatočnú praktickú výučbu medicíny.

\begin{itemize}
\item v medicíne je efektívne vzdelávanie dôležitejšie než kdekoľvek inde\\
	\begin{itemize}
	\item gamifikácia tomu vie dopomôcť
	\end{itemize}
\item praktická výučba musí byť dostatočná, aby medici dokázali povolanie vykonávať
	\begin{itemize}
	\item v tomto vie vo veľkej miere pomôcť VR
	\end{itemize}
\end{itemize}
\end{frame}



\section{Gamifikácia v medicínskom vzdelávaní}

\begin{frame}[fragile=singleslide]\frametitle{Gamifikácia v medicínskom vzdelávaní}
Ako už bolo spomenuté, gamifikácia využíva aplikáciu hernej mechaniky, estetiky a herného myslenia na motiváciu k činnosti, v tomto prípade na motiváciu k učeniu sa medicíny. V mobilnej či desktopovej aplikácii sú použité prvky ako skóre, odznaky za splnené ciele, či časové obmedzenia na naučenie sa daného učiva, ktoré prispievajú k efektívnemu sa učeniu. Ďalšou možnostou je odosielanie notifikácii po určitom čase od prvého naučenia sa nejakej témy, čo u študentov vyvolá tzv. "spaced repetition", ktorá predstavuje formu efektívneho memorovania. Dôležitú rolu zohráva aj bodovací systém, vďaka ktorému študenti vedia priebežne sledovať svoj pokrok, a taktiež je vďaka tomu možné zostaviť rebríček, kde sa medici vedia porovnať so svojimi rovesníkmi. 

\end{frame}

\section{Prax medikov a využitie serióznych hier}

\begin{frame}[fragile=singleslide]\frametitle{Prax medikov a využitie serióznych hier}


\includegraphics[scale=.15]{psik.jpg}

\begin{itemize}
\item psik na motivaciu\\
\end{itemize}

\end{frame}



\begin{frame}[fragile=singleslide]\frametitle{Zvýraznenie syntaxe}
\begin{itemize}
\item Na zvýraznenie syntaxe stačí použiť balík listings so správne nastaveným programovacím jazykom
\begin{lstlisting}
int na_druhu(int i) {
   return i * i;
}

int main() {
   printf("%d", na_druhu(118));
   return 0;
}
\end{lstlisting}

\item Jazyk C++ je ešte zaujímavejší: je multiparadigmový\footcite{\url{J. O. Coplien. Multi-Paradigm Design for C++. Addison-Wesley, 1998.}}
\end{itemize}
\end{frame}


\begin{frame}[fragile=singleslide]\frametitle{Rámiky}
\begin{poznamka}
Text možno uviesť v rámiku
\end{poznamka}

\begin{itemize}
\item Program

\begin{program}
\begin{lstlisting}
void main() {
   printf("%d", na_druhu(118));
}

void na_druhu(int i) {
   return i * i;
}
\end{lstlisting}
\end{program}

\item Výstup
\begin{vystup}
\begin{lstlisting}
13924
\end{lstlisting}
\end{vystup}

\end{itemize}
\end{frame}



\section*{Zhodnotenie a ďalšia práca}
% hviezdička zabezpečí, aby sa táto časť neocitla v prehľade prezentácie - každá prezentácia má zhodnotenie a prehľad by sa tým zbytočne zahlcoval

\begin{frame}[fragile=singleslide]\frametitle{Zhodnotenie a ďalšia práca}
\begin{itemize}
\item Každá prezentácia musí byť nejako uzavretá
\item Ale vždy je čo robiť ďalej\ldots{}
\end{itemize}
\end{frame}


\end{document}




Text \end{document} za príkazom \end{document} LaTeX ignoruje, takže tu môžete odkladať veci (aj celé slajdy), ktoré nechcete vymazať, lebo ich ešte možno budete potrebovať, avšak ich v danom momente nechcete mať v slajdoch.
